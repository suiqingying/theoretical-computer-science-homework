\documentclass[11pt, a4paper]{article}
\newcommand{\HomeworkHeader}{L01\_13-L01\_18}
% Shared LaTeX preamble for homework/*/main.tex

% --- Base layout ---
\usepackage[a4paper, top=2.5cm, bottom=2.5cm, left=2cm, right=2cm]{geometry}
\usepackage{fontspec}
\usepackage{xeCJK} % CJK fallback to avoid missing glyphs in Latin fonts
\usepackage{amsmath, amssymb, amsthm}
\usepackage{mathtools}
\usepackage{fancyhdr}
\usepackage{lastpage}
\usepackage[svgnames]{xcolor}
\usepackage{tikz}
\usetikzlibrary{automata, positioning, arrows.meta, bending, backgrounds}
\usepackage{pgfplots}
\pgfplotsset{compat=1.18}
\usepackage[most]{tcolorbox}
\usepackage{enumitem}

% --- Languages & fonts ---
\usepackage[english]{babel}
\babelprovide[import, main]{chinese}

\babelfont{rm}[UprightFont=*, BoldFont=* Bold, ItalicFont=* Italic, Scale=1.05]{Linux Libertine O}
\babelfont{sf}[UprightFont=*, BoldFont=* Bold, Scale=1.05]{Linux Biolinum O}
\babelfont[chinese]{rm}[
    UprightFont=*-Regular,
    BoldFont=*-Bold,
    ItalicFont=*-Regular,
    BoldItalicFont=*-Bold
]{Noto Serif CJK SC}
\babelfont[chinese]{sf}[
    UprightFont=*-Regular,
    BoldFont=*-Bold
]{Noto Sans CJK SC}
\babelfont{tt}{Noto Sans Mono}
\babelfont[chinese]{tt}{Noto Sans Mono CJK SC}

% xeCJK font setup (kept consistent with babelfont)
\setCJKmainfont{Noto Serif CJK SC}
\setCJKsansfont{Noto Sans CJK SC}
\setCJKmonofont{Noto Sans Mono CJK SC}

% --- Colors ---
\definecolor{primaryColor}{RGB}{46, 52, 64}
\definecolor{accentColor}{RGB}{94, 129, 172}
\definecolor{boxFill}{RGB}{236, 240, 241}
\definecolor{solLine}{RGB}{163, 190, 140}
\definecolor{expandColor}{RGB}{191, 97, 106}

% --- TikZ styles (DFA / NFA / PDA) ---
\tikzset{
    dfa_state/.style={
        state,
        thick,
        draw=accentColor,
        fill=boxFill,
        text=primaryColor,
        minimum size=1.1cm
    },
    dfa_edge/.style={
        ->,
        >=stealth,
        thick,
        draw=primaryColor,
        auto
    },
    pda_node/.style={
        state,
        thick,
        draw=accentColor,
        fill=boxFill,
        text=primaryColor,
        minimum size=1.2cm
    },
    pda_edge/.style={
        ->,
        >=stealth,
        thick,
        draw=primaryColor,
        auto
    },
    rule_expand/.style={
        pda_edge,
        draw=expandColor,
        text=expandColor
    },
    rule_match/.style={
        pda_edge,
        draw=solLine,
        text=solLine!80!black
    }
}

% --- Header / footer ---
\pagestyle{fancy}
\fancyhf{}
\lhead{\color{accentColor}\textbf{形式语言与自动机作业}}
\rhead{\color{primaryColor}\textbf{\HomeworkHeader}}
\cfoot{\small 第 \thepage \ 页,共 \pageref{LastPage} 页}
\setlength{\headheight}{24pt}
\addtolength{\topmargin}{-12pt}
\setlength{\emergencystretch}{2em}
\renewcommand{\headrulewidth}{0.4pt}
\renewcommand{\headrule}{\hbox to\headwidth{\color{accentColor}\leaders\hrule height \headrulewidth\hfill}}

% --- Problem box ---
\newtcolorbox[auto counter]{problem}[1][]{
    enhanced,
    breakable,
    colback=boxFill,
    colframe=accentColor,
    coltitle=white,
    fonttitle=\bfseries\large,
    title={题目 ~\thetcbcounter \quad #1},
    boxrule=0.5mm,
    arc=3mm,
    drop shadow=black!15!white,
    attach boxed title to top left={xshift=0.5cm, yshift*=-3mm},
    boxed title style={colback=accentColor}
}

% --- Solution environment ---
\newenvironment{solution}{
    \par\vspace{10pt}
    \noindent\textbf{\color{solLine} \Large \textit{Solution.}} \par
    \begingroup\color{primaryColor}\sloppy
    \leftskip1em \rightskip1em
    \noindent\rule{\textwidth}{0.4pt}
    \par\vspace{5pt}
}{
    \par\vspace{10pt}
    \noindent\rule[0.5ex]{\textwidth}{0.1pt}
    \endgroup
    \vspace{1.2cm}
}

% --- Title ---
\newcommand{\makeCustomTitle}[1]{
    \begin{center}
        \vspace*{1cm}
        {\Huge \bfseries \color{primaryColor} 作业 #1}\\
        \vspace{0.5cm}
        {\Large \color{accentColor} \today}
        \vspace{1.2cm}
    \end{center}
}


\begin{document}

\makeCustomTitle{L1.13 - L1.18:泵引理与 DFA 极小化}

% --- Problem 1 ---
\begin{problem}
给出下述语言的最小泵长度,并加以证明:
\[
0001^*,\quad 0^*1^*,\quad 001 \cup 0^*1^*,\quad 01^*,\quad \epsilon,\quad 1^*01^*01^*,\quad 1011,\quad \Sigma^*.
\]
\end{problem}

\begin{solution}
这里的“最小泵长度”按正则语言泵引理的定义取:对语言 $L$,最小的 $p\ge 1$ 使得对任意 $w\in L$ 且 $|w|\ge p$,
存在分解 $w=xyz$ 满足 $|xy|\le p$、$|y|\ge 1$ 且对所有 $i\ge 0$ 有 $xy^iz\in L$。

一般地,若一个 DFA 有 $n$ 个状态,则 $L$ 的一个泵长度可取 $p=n$。而“最小泵长度”往往等于最小 DFA 的状态数,但需逐个验证(这里给出可行的最小值与理由)。

\textbf{(a) $L=0001^*$:最小泵长度 $p=4$。}

最小 DFA 需要区分前缀是否已经读到恰好三个 0 以及是否已经进入尾部 $1^*$,共至少 4 个状态;因此 $p\ge 4$。取 $p=4$ 可泵:任取 $|w|\ge 4$ 的 $w\in L$,必形如 $0001^k$,取 $x=\epsilon,y=0,z=00 1^k$(或取 $y$ 为第 4 位的 1 当 $k\ge 1$),可检查 $xy^iz\in L$。

\textbf{(b) $L=0^*1^*$:最小泵长度 $p=2$。}

最小 DFA 有 3 个状态(处于 0 段、处于 1 段、陷阱),因此可取 $p=3$;但最小泵长度可更小。取 $p=2$:任取 $|w|\ge 2$ 且 $w\in 0^*1^*$,若 $w$ 以 0 开头取 $y=0$;否则以 1 开头取 $y=1$,均不会破坏“先 0 后 1”的结构,故可泵。

\textbf{(c) $L=001 \cup 0^*1^*$:最小泵长度 $p=2$。}

因为 $0^*1^*\subseteq L$,对长度 $\ge 2$ 的串,$L$ 与 $0^*1^*$ 的泵分解策略同样适用;特殊串 $001$ 长度为 3,也可取分解 $x=\epsilon,y=0,z=01$,$xy^iz=0^{i+1}1\in 0^*1^*\subseteq L$。因此 $p=2$ 可行,且 $p=1$ 不行(例如 $w=01$ 时 $|xy|\le 1$ 迫使 $y=0$,泵 $i=2$ 得 $001\notin 0^*1^*$ 且不等于 $001$,矛盾)。

\textbf{(d) $L=01^*$:最小泵长度 $p=2$。}

$p=1$ 不行($w=01$ 时 $|xy|\le 1$ 只能在首位 0 上泵,得 $001\notin L$)。取 $p=2$:对任意 $w=01^k,\,|w|\ge 2$,令 $x=0,y=1,z=1^{k-1}$,则 $xy^iz=01^{k-1+i}\in L$。

\textbf{(e) $L=\{\epsilon\}$:最小泵长度 $p=1$。}

因为不存在长度 $\ge 1$ 的串属于该语言,泵引理条件对所有 $w$ 真 vacuously true;按定义最小 $p$ 可取 1。

\textbf{(f) $L=1^*01^*01^*$:最小泵长度 $p=3$。}

该语言包含至少两个 0,最短非空串为 $00$(长度 2),但泵要求对 $|w|\ge p$ 才适用。取 $p=3$:任取 $w$,只要在前 3 位内找到一个 1 或 0 作为 $y$,泵不会改变“恰好两个 0 的相对顺序且允许任意多 1”的结构。$p=2$ 不行可用反例(例如 $w=010$ 的泵分解会在前两位强制泵到结构外)。

\textbf{(g) $L=\{1011\}$:最小泵长度 $p=5$。}

这是有限语言;对 $p\le 4$,串 $w=1011$ 满足 $|w|\ge p$,但任何 $|xy|\le p$ 的分解都会在泵 $i=0$ 或 $i=2$ 时改变长度,得到不在语言中的串,故不满足。取 $p=5$ 时条件 vacuously true(无 $|w|\ge 5$ 的串属于该语言)。

\textbf{(h) $L=\Sigma^*$:最小泵长度 $p=1$。}

任取 $|w|\ge 1$,令 $x=\epsilon,y$ 为首符号,$z$ 为剩余部分,则对所有 $i$,$xy^iz\in\Sigma^*$ 恒成立。
\end{solution}

% --- Problem 2 ---
\begin{problem}
使用泵引理证明下述语言不是正则的:
\[
A_1=\{0^n1^n2^n\mid n\ge 0\},\quad
A_2=\{www\mid w\in\{a,b\}^*\},\quad
A_3=\{a^{2^n}\mid n\ge 0\}.
\]
\end{problem}

\begin{solution}
\textbf{(1) $A_1$ 非正则}

假设 $A_1$ 正则,设泵长度为 $p$。取
\[
w=0^p1^p2^p\in A_1.
\]
任意分解 $w=xyz$ 满足 $|xy|\le p$,则 $y$ 只包含 0(即 $y=0^t,t\ge 1$)。泵 $i=0$ 得
\[
xz=0^{p-t}1^p2^p,
\]
0 的个数与 1、2 的个数不相等,故 $xz\notin A_1$,矛盾。

\textbf{(2) $A_2$ 非正则}

假设 $A_2$ 正则,泵长度为 $p$。取 $w=a^pb^pa^pb^pa^pb^p$,令 $u=a^pb^p$,则 $w=uuu\in A_2$。
任意分解 $w=xyz$ 且 $|xy|\le p$ 时,$y$ 只含 $a$(在第一个 $a^p$ 内)。泵 $i=2$ 得到前段 $a$ 个数改变,整串无法写成同一个串的三次重复(因为三个块必须长度相等且内容相同),与 $A_2$ 定义矛盾。

\textbf{(3) $A_3$ 非正则}

假设 $A_3$ 正则,泵长度为 $p$。取 $w=a^{2^p}\in A_3$。
任意分解 $w=xyz$ 且 $|xy|\le p$ 时,$y=a^t$,$1\le t\le p$。泵 $i=2$ 得
\[
xy^2z=a^{2^p+t}.
\]
但 $2^p < 2^p+t < 2^{p+1}$,因此 $2^p+t$ 不是 2 的幂,故 $xy^2z\notin A_3$,矛盾。
\end{solution}

% --- Problem 3 ---
\begin{problem}
考察语言
\[
F=\{a^ib^jc^k \mid i,j,k\ge 0,\ \text{且若 }i=1,\text{ 则 }j=k\}.
\]
\begin{enumerate}[label=(\alph*)]
    \item 证明:$F$ 不是正则的
    \item 证明:对于给定泵长度 $p$,$F$ 满足泵引理的三个条件
    \item 解释为什么上述两个结论与泵引理不矛盾
\end{enumerate}
\end{problem}

\begin{solution}
\textbf{(a) $F$ 非正则}

取子语言 $F' = F \cap a b^* c^*$(即限制 $i=1$),则
\[
F'=\{ab^jc^k \mid j=k\}=\{ab^nc^n\mid n\ge 0\},
\]
这是经典非正则语言,因此若 $F$ 正则则其与正则语言的交仍正则,矛盾。故 $F$ 非正则。

\textbf{(b) 对任意给定 $p$,$F$ 满足泵引理条件}

泵引理是“正则 $\Rightarrow$ 可泵”的必要条件,但“可泵”并不推出正则。这里要展示:对任意给定 $p$,对每个 $|w|\ge p$ 的 $w\in F$,\emph{确实存在某个分解}满足泵条件。

设 $w=a^ib^jc^k\in F$ 且 $|w|\ge p$。只需给出一种分解方案:
\begin{itemize}
    \item 若 $i\ge 1$,取 $x=\epsilon,\ y=a,\ z=a^{i-1}b^jc^k$。则对所有 $t\ge 0$,
    \[
    xy^tz = a^t a^{i-1}b^jc^k = a^{t+i-1}b^jc^k.
    \]
    当 $t+i-1=1$ 时(即 $t=2-i$,只可能在 $i=1$ 且 $t=1$),原串满足 $j=k$;其余情况下 $a$ 的指数 $\ne 1$,语言对 $j,k$ 无额外约束,因此 $xy^tz\in F$。
    \item 若 $i=0$,则 $w=b^jc^k$ 自动满足 $i\ne 1$ 的情形。取 $x=\epsilon,\ y=b,\ z=b^{j-1}c^k$(若 $j=0$ 则取 $y=c$),泵动不会引入 $i=1$,故仍在 $F$。
\end{itemize}
因此对任意给定 $p$,$F$ 满足泵引理的三个形式条件。

\textbf{(c) 不矛盾的原因}

泵引理是\textbf{正则语言的必要条件}而非充分条件:存在非正则语言也“满足”泵引理形式条件(即对某个 $p$,每个足够长的串都能找到某种可泵分解)。
因此“$F$ 非正则”与“对任意给定 $p$,$F$ 仍可满足泵条件”并不矛盾。
\end{solution}

% --- Problem 4 ---
\begin{problem}
设 $L=\{1^q \mid q \text{ 为素数}\}$,用泵引理证明该语言不是正则的。
\end{problem}

\begin{solution}
假设 $L$ 正则,设泵长度为 $p$。取一个素数 $q>p$,并令 $w=1^q\in L$。
任意分解 $w=xyz$ 满足 $|xy|\le p$,则 $y=1^t$,其中 $1\le t\le p$。
泵 $i=q+1$ 得
\[
xy^{q+1}z = 1^{q + qt} = 1^{q(1+t)}.
\]
指数 $q(1+t)$ 为合数($q\ge 2$ 且 $1+t\ge 2$),因此 $xy^{q+1}z\notin L$,与泵引理矛盾。故 $L$ 非正则。
\end{solution}

% --- Problem 5 ---
\begin{problem}
极小化如下 DFA(两题,题面给出了状态图)。
\end{problem}

\begin{solution}
\textbf{(1) 字母表 $\{a,b\}$ 的 DFA(状态 $A,B,C,D,E$,终态 $E$)}

由题图读取转移:
\[
\begin{array}{c|cc}
 & a & b\\\hline
A & B & C\\
B & B & D\\
C & B & C\\
D & B & E\\
E & B & C\\
\end{array}
\]
初态为 $A$,终态集合 $F=\{E\}$。

\textbf{等价类划分(填表法同理):}
初分割 $P_0=\{\{E\},\{A,B,C,D\}\}$。
在 $\{A,B,C,D\}$ 内,$D$ 在输入 $b$ 下转入终态 $E$,其余不转入,故分裂为
\[
P_1=\{\{E\},\{D\},\{A,B,C\}\}.
\]
再看 $\{A,B,C\}$:在输入 $b$ 下,$B\to D$,而 $A,C$ 都留在 $\{A,B,C\}$,故再分裂为
\[
P_2=\{\{E\},\{D\},\{B\},\{A,C\}\}.
\]
最后检查 $\{A,C\}$:两者在 $a$ 下都到 $B$,在 $b$ 下都回到 $\{A,C\}$,因此等价。

所以最小 DFA 有 4 个状态:$[AC],[B],[D],[E]$,初态为 $[AC]$,终态为 $[E]$。
其转移为:
\[
\begin{array}{c|cc}
 & a & b\\\hline
\lbrack AC\rbrack & \lbrack B\rbrack & \lbrack AC\rbrack\\
\lbrack B\rbrack & \lbrack B\rbrack & \lbrack D\rbrack\\
\lbrack D\rbrack & \lbrack B\rbrack & \lbrack E\rbrack\\
\lbrack E\rbrack & \lbrack B\rbrack & \lbrack AC\rbrack\\
\end{array}
\]

\textbf{(2) 字母表 $\{0,1\}$ 的 DFA(状态 $Q_0,\dots,Q_5$,终态 $Q_1,Q_2,Q_5$)}

由题图读取转移:
\[
\begin{array}{c|cc}
 & 0 & 1\\\hline
Q_0 & Q_1 & Q_2\\
Q_1 & Q_3 & Q_4\\
Q_2 & Q_4 & Q_3\\
Q_3 & Q_5 & Q_5\\
Q_4 & Q_5 & Q_5\\
Q_5 & Q_5 & Q_5\\
\end{array}
\]
终态集合 $F=\{Q_1,Q_2,Q_5\}$。

初分割:
\[
P_0=\{\{Q_1,Q_2,Q_5\},\{Q_0,Q_3,Q_4\}\}.
\]
在终态类中,$Q_5$ 在 $0/1$ 下都回到自身,而 $Q_1,Q_2$ 都会转入非终态类,故
\[
P_1=\{\{Q_5\},\{Q_1,Q_2\},\{Q_0,Q_3,Q_4\}\}.
\]
在 $\{Q_0,Q_3,Q_4\}$ 中,$Q_3,Q_4$ 都在 $0/1$ 下转入 $Q_5$,而 $Q_0$ 转入 $\{Q_1,Q_2\}$,因此
\[
P_2=\{\{Q_5\},\{Q_1,Q_2\},\{Q_3,Q_4\},\{Q_0\}\}.
\]
检查可知 $\{Q_1,Q_2\}$ 与 $\{Q_3,Q_4\}$ 都不再分裂。

故最小 DFA 的 4 个状态可记为 $[Q_0],[Q_{12}],[Q_{34}],[Q_5]$,初态 $[Q_0]$,终态为 $[Q_{12}]$ 与 $[Q_5]$。
转移为:
\[
\begin{array}{c|cc}
 & 0 & 1\\\hline
\lbrack Q_0\rbrack & \lbrack Q_{12}\rbrack & \lbrack Q_{12}\rbrack\\
\lbrack Q_{12}\rbrack & \lbrack Q_{34}\rbrack & \lbrack Q_{34}\rbrack\\
\lbrack Q_{34}\rbrack & \lbrack Q_5\rbrack & \lbrack Q_5\rbrack\\
\lbrack Q_5\rbrack & \lbrack Q_5\rbrack & \lbrack Q_5\rbrack\\
\end{array}
\]
\end{solution}

% --- Problem 6 ---
\begin{problem}
利用迈希尔-尼罗德定理证明若干语言是否正则(题面列出 $A_1,A_2,A_3$ 以及关于 $A,B$ 的结论)。
\end{problem}

\begin{solution}
题面(见 \texttt{assets/problem\_L01\_13-L01\_18.pdf})给出:
\[
\begin{aligned}
A_1&=\{0^n1^n2^n\mid n\ge 0\},\\
A_2&=\{www\mid w\in\{a,b\}^*\},\\
A_3&=\{a^{2^n}\mid n\ge 0\},\\
\Sigma&=\{0,1\},\quad
A=\{0^ku0^k\mid k\ge 1,\ u\in\Sigma^*\},\quad
B=\{0^k1u0^k\mid k\ge 1,\ u\in\Sigma^*\}.
\end{aligned}
\]
用迈希尔-尼罗德定理:若语言正则,则其不可区分关系 $\equiv_L$ 只有有限多个等价类;反之若能构造无限多个两两可区分串,则语言非正则。

\textbf{(1) $A_1$ 不是正则语言}

取集合 $X=\{0^n\mid n\ge 0\}$。对任意 $m\ne n$,令后缀 $z_n=1^n2^n$,
则
\[
0^n z_n = 0^n1^n2^n\in A_1,\qquad
0^m z_n = 0^m1^n2^n\notin A_1.
\]
因此 $0^m$ 与 $0^n$ 可区分,$\equiv_{A_1}$ 有无限多个等价类,故 $A_1$ 非正则。

\textbf{(2) $A_2$ 不是正则语言}

取 $x_n=a^n$。令后缀
\[
z_n=b\,a^n\,b\,a^n\,b.
\]
则
\[
x_n z_n = a^n b a^n b a^n b = (a^n b)(a^n b)(a^n b)\in A_2.
\]
对任意 $m\ne n$,
\[
x_m z_n = a^m b a^n b a^n b
\]
若它属于 $A_2$,则必可写为 $uuu$。由于串中恰有 3 个字母 $b$,可推出 $u$ 中恰有 1 个 $b$,且 $u$ 必以 $b$ 结尾,
因此 $u$ 只能形如 $a^t b$。于是 $uuu$ 的三段 $a$ 的长度必须相同,
与本串三段 $a$ 的长度为 $m,n,n$(且 $m\ne n$)矛盾。故 $x_m z_n\notin A_2$。
因此 $\{a^n\}$ 两两可区分,$A_2$ 非正则。

\textbf{(3) $A_3$ 不是正则语言}

取 $x_n=a^{2^n}$。令后缀 $z_n=a^{2^n}$,则
\[
x_n z_n = a^{2^n+2^n}=a^{2^{n+1}}\in A_3.
\]
若 $m\ne n$,则 $2^m+2^n$ 不是 2 的幂,因此
\[
x_m z_n = a^{2^m+2^n}\notin A_3.
\]
所以 $\equiv_{A_3}$ 有无限多个等价类,$A_3$ 非正则。

\textbf{(4) $A$ 是正则语言}

注意到对任意 $w\in A$,$w$ 至少以一个 $0$ 开头且以一个 $0$ 结尾;反之,任意以 $0$ 开头且以 $0$ 结尾的串都可取 $k=1$ 写成 $0^k u 0^k$。
因此
\[
A = 0\Sigma^*0,
\]
是正则语言(正则表达式即 $0(0+1)^*0$)。

\textbf{(5) $B$ 不是正则语言}

取串集 $x_k=0^k1\ (k\ge 1)$。对任意 $i<j$,取后缀 $z=0^i$,则
\[
x_i z = 0^i10^i \in B\quad(\text{取 }u=\epsilon),
\]
但
\[
x_j z = 0^j10^i \notin B
\]
因为其第一个 $1$ 出现在 $j$ 个 $0$ 之后,若按 $B$ 的形式必须以 $0^j1$ 开头并以 $0^j$ 结尾,但该串只以 $0^i$ 结尾($i<j$)。
因此 $\{x_k\}$ 两两可区分,$\equiv_B$ 有无限多个等价类,故 $B$ 非正则。
\end{solution}

\end{document}
