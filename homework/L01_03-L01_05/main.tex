\documentclass[11pt, a4paper]{article}
\newcommand{\HomeworkHeader}{L01\_03-L01\_05}
% Shared LaTeX preamble for homework/*/main.tex

% --- Base layout ---
\usepackage[a4paper, top=2.5cm, bottom=2.5cm, left=2cm, right=2cm]{geometry}
\usepackage{fontspec}
\usepackage{xeCJK} % CJK fallback to avoid missing glyphs in Latin fonts
\usepackage{amsmath, amssymb, amsthm}
\usepackage{mathtools}
\usepackage{fancyhdr}
\usepackage{lastpage}
\usepackage[svgnames]{xcolor}
\usepackage{tikz}
\usetikzlibrary{automata, positioning, arrows.meta, bending, backgrounds}
\usepackage{pgfplots}
\pgfplotsset{compat=1.18}
\usepackage[most]{tcolorbox}
\usepackage{enumitem}

% --- Languages & fonts ---
\usepackage[english]{babel}
\babelprovide[import, main]{chinese}

\babelfont{rm}[UprightFont=*, BoldFont=* Bold, ItalicFont=* Italic, Scale=1.05]{Linux Libertine O}
\babelfont{sf}[UprightFont=*, BoldFont=* Bold, Scale=1.05]{Linux Biolinum O}
\babelfont[chinese]{rm}[
    UprightFont=*-Regular,
    BoldFont=*-Bold,
    ItalicFont=*-Regular,
    BoldItalicFont=*-Bold
]{Noto Serif CJK SC}
\babelfont[chinese]{sf}[
    UprightFont=*-Regular,
    BoldFont=*-Bold
]{Noto Sans CJK SC}
\babelfont{tt}{Noto Sans Mono}
\babelfont[chinese]{tt}{Noto Sans Mono CJK SC}

% xeCJK font setup (kept consistent with babelfont)
\setCJKmainfont{Noto Serif CJK SC}
\setCJKsansfont{Noto Sans CJK SC}
\setCJKmonofont{Noto Sans Mono CJK SC}

% --- Colors ---
\definecolor{primaryColor}{RGB}{46, 52, 64}
\definecolor{accentColor}{RGB}{94, 129, 172}
\definecolor{boxFill}{RGB}{236, 240, 241}
\definecolor{solLine}{RGB}{163, 190, 140}
\definecolor{expandColor}{RGB}{191, 97, 106}

% --- TikZ styles (DFA / NFA / PDA) ---
\tikzset{
    dfa_state/.style={
        state,
        thick,
        draw=accentColor,
        fill=boxFill,
        text=primaryColor,
        minimum size=1.1cm
    },
    dfa_edge/.style={
        ->,
        >=stealth,
        thick,
        draw=primaryColor,
        auto
    },
    pda_node/.style={
        state,
        thick,
        draw=accentColor,
        fill=boxFill,
        text=primaryColor,
        minimum size=1.2cm
    },
    pda_edge/.style={
        ->,
        >=stealth,
        thick,
        draw=primaryColor,
        auto
    },
    rule_expand/.style={
        pda_edge,
        draw=expandColor,
        text=expandColor
    },
    rule_match/.style={
        pda_edge,
        draw=solLine,
        text=solLine!80!black
    }
}

% --- Header / footer ---
\pagestyle{fancy}
\fancyhf{}
\lhead{\color{accentColor}\textbf{形式语言与自动机作业}}
\rhead{\color{primaryColor}\textbf{\HomeworkHeader}}
\cfoot{\small 第 \thepage \ 页,共 \pageref{LastPage} 页}
\setlength{\headheight}{24pt}
\addtolength{\topmargin}{-12pt}
\setlength{\emergencystretch}{2em}
\renewcommand{\headrulewidth}{0.4pt}
\renewcommand{\headrule}{\hbox to\headwidth{\color{accentColor}\leaders\hrule height \headrulewidth\hfill}}

% --- Problem box ---
\newtcolorbox[auto counter]{problem}[1][]{
    enhanced,
    breakable,
    colback=boxFill,
    colframe=accentColor,
    coltitle=white,
    fonttitle=\bfseries\large,
    title={题目 ~\thetcbcounter \quad #1},
    boxrule=0.5mm,
    arc=3mm,
    drop shadow=black!15!white,
    attach boxed title to top left={xshift=0.5cm, yshift*=-3mm},
    boxed title style={colback=accentColor}
}

% --- Solution environment ---
\newenvironment{solution}{
    \par\vspace{10pt}
    \noindent\textbf{\color{solLine} \Large \textit{Solution.}} \par
    \begingroup\color{primaryColor}\sloppy
    \leftskip1em \rightskip1em
    \noindent\rule{\textwidth}{0.4pt}
    \par\vspace{5pt}
}{
    \par\vspace{10pt}
    \noindent\rule[0.5ex]{\textwidth}{0.1pt}
    \endgroup
    \vspace{1.2cm}
}

% --- Title ---
\newcommand{\makeCustomTitle}[1]{
    \begin{center}
        \vspace*{1cm}
        {\Huge \bfseries \color{primaryColor} 作业 #1}\\
        \vspace{0.5cm}
        {\Large \color{accentColor} \today}
        \vspace{1.2cm}
    \end{center}
}


\begin{document}

\makeCustomTitle{L1.3 - L1.5:NFA/DFA 转换与 $\epsilon$-消除}

% --- Problem 1 ---
\begin{problem}
将如下 NFA 转换成等价的 DFA:
\begin{enumerate}[label=(\alph*)]
  \item 字母表 $\Sigma=\{0,1\}$,状态 $q_1$ 为初态,$q_3$ 为终态。转移如图所示:$q_1$ 上有 $0,1$ 自环,且 $q_1\xrightarrow{1}q_2$,$q_2\xrightarrow{0,1}q_3$。
  \item 字母表 $\Sigma=\{a,b,c\}$,初态 $A$,终态 $B$。转移如图所示:$A\xrightarrow{\epsilon,c}B$,$A\xrightarrow{a,c}C$,$C\xrightarrow{\epsilon,b}A$,$C\xrightarrow{b}B$,$C\xrightarrow{c}C$。
\end{enumerate}
\end{problem}

\begin{solution}
\textbf{(a) 子集构造法}

NFA 的转移:
\[
\delta(q_1,0)=\{q_1\},\quad \delta(q_1,1)=\{q_1,q_2\},\quad
\delta(q_2,0)=\{q_3\},\quad \delta(q_2,1)=\{q_3\},\quad
\delta(q_3,0)=\delta(q_3,1)=\emptyset.
\]
构造 DFA 的状态为 NFA 状态集合:
\[
S_0=\{q_1\}\ (\text{初态}),\quad
S_1=\{q_1,q_2\},\quad
S_2=\{q_1,q_3\},\quad
S_3=\{q_1,q_2,q_3\}.
\]
接受态为包含 $q_3$ 的集合:$S_2,S_3$。

转移表:
\[
\begin{array}{c|cc}
\text{DFA 状态} & 0 & 1\\\hline
S_0=\{q_1\} & S_0 & S_1\\
S_1=\{q_1,q_2\} & S_2 & S_3\\
S_2=\{q_1,q_3\} & S_0 & S_1\\
S_3=\{q_1,q_2,q_3\} & S_2 & S_3\\
\end{array}
\]

\textbf{(b) 含 $\epsilon$ 的子集构造法}

先计算 $\epsilon$-闭包:
\[
\epsilon\text{-cl}(A)=\{A,B\},\quad \epsilon\text{-cl}(B)=\{B\},\quad
\epsilon\text{-cl}(C)=\{A,B,C\}\ (\text{因 }C\xrightarrow{\epsilon}A,\ A\xrightarrow{\epsilon}B).
\]
DFA 初态为 $\epsilon\text{-cl}(\{A\})=\{A,B\}$,并且凡是包含 $B$ 的集合都是接受态(因为可通过 $\epsilon$ 到达终态)。

可达的 DFA 状态只有三个:
\[
T_0=\{A,B\}\ (\text{初态,接受}),\quad
T_1=\{A,B,C\}\ (\text{接受}),\quad
T_\emptyset=\emptyset\ (\text{陷阱态,非接受}).
\]
逐字母计算转移($\delta_D(X,x)=\epsilon\text{-cl}(\text{move}(X,x))$),得到:
\[
\begin{array}{c|ccc}
\text{DFA 状态} & a & b & c\\\hline
T_0=\{A,B\} & T_1 & T_\emptyset & T_1\\
T_1=\{A,B,C\} & T_1 & T_0 & T_1\\
T_\emptyset=\emptyset & T_\emptyset & T_\emptyset & T_\emptyset\\
\end{array}
\]
\end{solution}

% --- Problem 2 ---
\begin{problem}
$\Sigma=\{a,b\}$,设计识别“以 \texttt{abb} 结尾”的 NFA,并将其转换成等价 DFA。
\end{problem}

\begin{solution}
\textbf{NFA 设计(允许回退重新匹配)}

取状态集合 $\{s_0,s_1,s_2,s_3\}$,其中 $s_0$ 为初态,$s_3$ 为终态。转移定义为:
\[
\begin{aligned}
\delta(s_0,a)&=\{s_0,s_1\}, & \delta(s_0,b)&=\{s_0\},\\
\delta(s_1,b)&=\{s_2\}, & \delta(s_1,a)&=\emptyset,\\
\delta(s_2,b)&=\{s_3\}, & \delta(s_2,a)&=\emptyset,\\
\delta(s_3,a)&=\emptyset, & \delta(s_3,b)&=\emptyset.
\end{aligned}
\]
直观上:$s_1,s_2,s_3$ 分别表示“刚看到后缀的 \texttt{a}、\texttt{ab}、\texttt{abb}”。

\textbf{等价 DFA(最短后缀法 / 子集构造的最小结果)}

用 4 个状态表示“当前已匹配的最长后缀是 \texttt{abb} 的前缀”的长度:
\[
q_0:\epsilon,\quad q_1:\texttt{a},\quad q_2:\texttt{ab},\quad q_3:\texttt{abb}(\text{接受}).
\]
转移表:
\[
\begin{array}{c|cc}
 & a & b\\\hline
q_0 & q_1 & q_0\\
q_1 & q_1 & q_2\\
q_2 & q_1 & q_3\\
q_3 & q_1 & q_0\\
\end{array}
\]
终态只有 $q_3$,它恰好对应“读入串以 \texttt{abb} 结尾”。
\end{solution}

% --- Problem 3 ---
\begin{problem}
将题 1(b) 的 $\epsilon$-NFA 转换成等价的\textbf{不含 $\epsilon$-移动}的 NFA。
\end{problem}

\begin{solution}
对每个状态 $p$,令 $\epsilon$-闭包为 $\epsilon\text{-cl}(p)$。消 $\epsilon$ 的标准做法是:
\[
\delta'(p,x)=\epsilon\text{-cl}\Big(\bigcup_{r\in \epsilon\text{-cl}(p)}\delta(r,x)\Big),\qquad
F'=\{p\mid \epsilon\text{-cl}(p)\cap F\ne\emptyset\}.
\]
本题中 $F=\{B\}$,且
\[
\epsilon\text{-cl}(A)=\{A,B\},\quad \epsilon\text{-cl}(B)=\{B\},\quad \epsilon\text{-cl}(C)=\{A,B,C\},
\]
因此新终态为 $F'=\{A,B,C\}$(初态 $A$ 也成为终态,对应原自动机可通过 $\epsilon$ 直接到达 $B$)。

计算 $\delta'$:
\[
\begin{array}{c|ccc}
\delta' & a & b & c\\\hline
A & \{A,B,C\} & \emptyset & \{A,B,C\}\\
B & \emptyset & \emptyset & \emptyset\\
C & \{A,B,C\} & \{A,B\} & \{A,B,C\}\\
\end{array}
\]
该 NFA 不含 $\epsilon$-边,且与原 $\epsilon$-NFA 等价。
\end{solution}

\end{document}

