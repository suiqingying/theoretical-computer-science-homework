\documentclass[11pt, a4paper]{article}
\newcommand{\HomeworkHeader}{L02\_01-L02\_03}
% Shared LaTeX preamble for homework/*/main.tex

% --- Base layout ---
\usepackage[a4paper, top=2.5cm, bottom=2.5cm, left=2cm, right=2cm]{geometry}
\usepackage{fontspec}
\usepackage{xeCJK} % CJK fallback to avoid missing glyphs in Latin fonts
\usepackage{amsmath, amssymb, amsthm}
\usepackage{mathtools}
\usepackage{fancyhdr}
\usepackage{lastpage}
\usepackage[svgnames]{xcolor}
\usepackage{tikz}
\usetikzlibrary{automata, positioning, arrows.meta, bending, backgrounds}
\usepackage{pgfplots}
\pgfplotsset{compat=1.18}
\usepackage[most]{tcolorbox}
\usepackage{enumitem}

% --- Languages & fonts ---
\usepackage[english]{babel}
\babelprovide[import, main]{chinese}

\babelfont{rm}[UprightFont=*, BoldFont=* Bold, ItalicFont=* Italic, Scale=1.05]{Linux Libertine O}
\babelfont{sf}[UprightFont=*, BoldFont=* Bold, Scale=1.05]{Linux Biolinum O}
\babelfont[chinese]{rm}[
    UprightFont=*-Regular,
    BoldFont=*-Bold,
    ItalicFont=*-Regular,
    BoldItalicFont=*-Bold
]{Noto Serif CJK SC}
\babelfont[chinese]{sf}[
    UprightFont=*-Regular,
    BoldFont=*-Bold
]{Noto Sans CJK SC}
\babelfont{tt}{Noto Sans Mono}
\babelfont[chinese]{tt}{Noto Sans Mono CJK SC}

% xeCJK font setup (kept consistent with babelfont)
\setCJKmainfont{Noto Serif CJK SC}
\setCJKsansfont{Noto Sans CJK SC}
\setCJKmonofont{Noto Sans Mono CJK SC}

% --- Colors ---
\definecolor{primaryColor}{RGB}{46, 52, 64}
\definecolor{accentColor}{RGB}{94, 129, 172}
\definecolor{boxFill}{RGB}{236, 240, 241}
\definecolor{solLine}{RGB}{163, 190, 140}
\definecolor{expandColor}{RGB}{191, 97, 106}

% --- TikZ styles (DFA / NFA / PDA) ---
\tikzset{
    dfa_state/.style={
        state,
        thick,
        draw=accentColor,
        fill=boxFill,
        text=primaryColor,
        minimum size=1.1cm
    },
    dfa_edge/.style={
        ->,
        >=stealth,
        thick,
        draw=primaryColor,
        auto
    },
    pda_node/.style={
        state,
        thick,
        draw=accentColor,
        fill=boxFill,
        text=primaryColor,
        minimum size=1.2cm
    },
    pda_edge/.style={
        ->,
        >=stealth,
        thick,
        draw=primaryColor,
        auto
    },
    rule_expand/.style={
        pda_edge,
        draw=expandColor,
        text=expandColor
    },
    rule_match/.style={
        pda_edge,
        draw=solLine,
        text=solLine!80!black
    }
}

% --- Header / footer ---
\pagestyle{fancy}
\fancyhf{}
\lhead{\color{accentColor}\textbf{形式语言与自动机作业}}
\rhead{\color{primaryColor}\textbf{\HomeworkHeader}}
\cfoot{\small 第 \thepage \ 页,共 \pageref{LastPage} 页}
\setlength{\headheight}{24pt}
\addtolength{\topmargin}{-12pt}
\setlength{\emergencystretch}{2em}
\renewcommand{\headrulewidth}{0.4pt}
\renewcommand{\headrule}{\hbox to\headwidth{\color{accentColor}\leaders\hrule height \headrulewidth\hfill}}

% --- Problem box ---
\newtcolorbox[auto counter]{problem}[1][]{
    enhanced,
    breakable,
    colback=boxFill,
    colframe=accentColor,
    coltitle=white,
    fonttitle=\bfseries\large,
    title={题目 ~\thetcbcounter \quad #1},
    boxrule=0.5mm,
    arc=3mm,
    drop shadow=black!15!white,
    attach boxed title to top left={xshift=0.5cm, yshift*=-3mm},
    boxed title style={colback=accentColor}
}

% --- Solution environment ---
\newenvironment{solution}{
    \par\vspace{10pt}
    \noindent\textbf{\color{solLine} \Large \textit{Solution.}} \par
    \begingroup\color{primaryColor}\sloppy
    \leftskip1em \rightskip1em
    \noindent\rule{\textwidth}{0.4pt}
    \par\vspace{5pt}
}{
    \par\vspace{10pt}
    \noindent\rule[0.5ex]{\textwidth}{0.1pt}
    \endgroup
    \vspace{1.2cm}
}

% --- Title ---
\newcommand{\makeCustomTitle}[1]{
    \begin{center}
        \vspace*{1cm}
        {\Huge \bfseries \color{primaryColor} 作业 #1}\\
        \vspace{0.5cm}
        {\Large \color{accentColor} \today}
        \vspace{1.2cm}
    \end{center}
}


\begin{document}

\makeCustomTitle{L2.1 - L2.3:正则文法与语法范畴}

% --- Problem 1 ---
\begin{problem}
构造一个 DFA 接受如下文法定义的语言:
\[
S \to abA,\qquad
A \to baB,\qquad
B \to aA \mid bb.
\]
\end{problem}

\begin{solution}
该右线性文法生成的串必以 $S\Rightarrow abA\Rightarrow abbaB$ 开头,
在 $B$ 处要么选择 $bb$ 结束,要么选择 $aA$ 并再次经过 $A\Rightarrow baB$ 回到 $B$。
因此语言可写为:
\[
L = \texttt{abba}(\texttt{aba})^*\texttt{bb}.
\]

据此构造 DFA(含陷阱态 $\bot$)。令状态表示“已匹配到的固定前缀/循环位置”:
\[
q_0 \xrightarrow{a} q_1 \xrightarrow{b} q_2 \xrightarrow{b} q_3 \xrightarrow{a} q_4
\]
其中 $q_4$ 表示已经读到 \texttt{abba} 且当前位于循环入口(对应非终结符 $B$ 的位置)。
循环块 \texttt{aba} 由 $q_4 \xrightarrow{a} q_5 \xrightarrow{b} q_6 \xrightarrow{a} q_4$ 实现;
终止块 \texttt{bb} 由 $q_4 \xrightarrow{b} q_7 \xrightarrow{b} q_8$ 实现,其中 $q_8$ 为唯一接受态。
所有未定义的转移均进入陷阱态 $\bot$(并在 $\bot$ 上自环)。

完整转移(字母表 $\{a,b\}$):
\[
\begin{array}{c|cc}
 & a & b\\\hline
q_0 & q_1 & \bot\\
q_1 & \bot & q_2\\
q_2 & \bot & q_3\\
q_3 & q_4 & \bot\\
q_4 & q_5 & q_7\\
q_5 & \bot & q_6\\
q_6 & q_4 & \bot\\
q_7 & \bot & q_8\\
q_8 & \bot & \bot\\
\bot & \bot & \bot\\
\end{array}
\]
该 DFA 恰好接受 $\texttt{abba}(\texttt{aba})^*\texttt{bb}$。
\end{solution}

% --- Problem 2 ---
\begin{problem}
构造产生语言 $L(aa^*(ab+a)^*)$ 的正则文法。
\end{problem}

\begin{solution}
正则表达式 $aa^*(ab+a)^*$ 等价于 $a^+(ab\mid a)^*$(至少一个 $a$,之后由若干个 \texttt{a} 或 \texttt{ab} 拼接)。
给出一个右线性文法(正则文法):
\[
\boxed{
\begin{aligned}
S &\to aS \mid aR,\\
R &\to aR \mid aB \mid \epsilon,\\
B &\to bR.
\end{aligned}}
\]
其中 $S$ 保证至少产生一个 $a$;在 $R$ 中可以选择产生 \texttt{a}(用 $R\to aR$)或产生 \texttt{ab}(用 $R\to aB\to abR$),并可用 $R\to\epsilon$ 结束。
\end{solution}

% --- Problem 3 ---
\begin{problem}
构造产生语言 $L((aab^*ab)^*)$ 的右线性文法。
\end{problem}

\begin{solution}
一个块为 $\texttt{aa} \,\texttt{b}^*\, \texttt{ab}$,整体是其 Kleene 星,可产生空串。
给出右线性文法:
\[
\boxed{
\begin{aligned}
S &\to \epsilon \mid aA,\\
A &\to aB,\\
B &\to bB \mid aC,\\
C &\to bS.
\end{aligned}}
\]
推导:$S\Rightarrow aA\Rightarrow aaB\Rightarrow aab^*aC\Rightarrow aab^*abS$,回到 $S$ 继续重复或用 $\epsilon$ 结束。
\end{solution}

% --- Problem 4 ---
\begin{problem}
设文法 $G$ 的产生式如下,给出 $G$ 的每个语法范畴(非终结符)代表的集合:
\[
\begin{aligned}
S&\to aSa \mid aaSaa \mid aAa\\
A&\to bA \mid bbbA \mid bB\\
B&\to cB \mid cC\\
C&\to ccC \mid DD\\
D&\to dD \mid d
\end{aligned}
\]
\end{problem}

\begin{solution}
\textbf{(1) $D$:}
\[
D\to dD\mid d \quad\Rightarrow\quad L(D)=\{d^n\mid n\ge 1\}.
\]

\textbf{(2) $C$:}
\[
C\to ccC\mid DD.
\]
其中 $DD$ 连接两个 $d^+$,因此 $L(DD)=\{d^n\mid n\ge 2\}$。再在前面重复添加 \texttt{cc},得
\[
L(C)=\{(cc)^k d^n \mid k\ge 0,\ n\ge 2\}=\{c^{2k}d^n\mid k\ge 0,\ n\ge 2\}.
\]

\textbf{(3) $B$:}
\[
B\to cB\mid cC \quad\Rightarrow\quad L(B)=\{c^m x\mid m\ge 1,\ x\in L(C)\}.
\]
因此
\[
L(B)=\{c^t d^n\mid t\ge 1,\ n\ge 2\}.
\]

\textbf{(4) $A$:}
\[
A\to bA\mid bbbA\mid bB \quad\Rightarrow\quad A\Rightarrow (b\mid bbb)^*\,bB.
\]
即先产生至少一个 \texttt{b},再进入 $B$,于是
\[
L(A)=\{b^m c^t d^n \mid m\ge 1,\ t\ge 1,\ n\ge 2\}.
\]

\textbf{(5) $S$:}
\[
S\to aSa\mid aaSaa\mid aAa
\]
表示在两端对称包裹 1 个或 2 个 \texttt{a},最终落到 $aAa$。因此
\[
L(S)=\{a^k\,x\,a^k \mid k\ge 1,\ x\in L(A)\}.
\]
代入 $L(A)$ 得
\[
\boxed{
L(S)=\{a^k b^m c^t d^n a^k \mid k\ge 1,\ m\ge 1,\ t\ge 1,\ n\ge 2\}.}
\]
\end{solution}

\end{document}

