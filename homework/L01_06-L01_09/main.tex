\documentclass[11pt, a4paper]{article}
\newcommand{\HomeworkHeader}{L01\_06-L01\_09}
% Shared LaTeX preamble for homework/*/main.tex

% --- Base layout ---
\usepackage[a4paper, top=2.5cm, bottom=2.5cm, left=2cm, right=2cm]{geometry}
\usepackage{fontspec}
\usepackage{xeCJK} % CJK fallback to avoid missing glyphs in Latin fonts
\usepackage{amsmath, amssymb, amsthm}
\usepackage{mathtools}
\usepackage{fancyhdr}
\usepackage{lastpage}
\usepackage[svgnames]{xcolor}
\usepackage{tikz}
\usetikzlibrary{automata, positioning, arrows.meta, bending, backgrounds}
\usepackage{pgfplots}
\pgfplotsset{compat=1.18}
\usepackage[most]{tcolorbox}
\usepackage{enumitem}

% --- Languages & fonts ---
\usepackage[english]{babel}
\babelprovide[import, main]{chinese}

\babelfont{rm}[UprightFont=*, BoldFont=* Bold, ItalicFont=* Italic, Scale=1.05]{Linux Libertine O}
\babelfont{sf}[UprightFont=*, BoldFont=* Bold, Scale=1.05]{Linux Biolinum O}
\babelfont[chinese]{rm}[
    UprightFont=*-Regular,
    BoldFont=*-Bold,
    ItalicFont=*-Regular,
    BoldItalicFont=*-Bold
]{Noto Serif CJK SC}
\babelfont[chinese]{sf}[
    UprightFont=*-Regular,
    BoldFont=*-Bold
]{Noto Sans CJK SC}
\babelfont{tt}{Noto Sans Mono}
\babelfont[chinese]{tt}{Noto Sans Mono CJK SC}

% xeCJK font setup (kept consistent with babelfont)
\setCJKmainfont{Noto Serif CJK SC}
\setCJKsansfont{Noto Sans CJK SC}
\setCJKmonofont{Noto Sans Mono CJK SC}

% --- Colors ---
\definecolor{primaryColor}{RGB}{46, 52, 64}
\definecolor{accentColor}{RGB}{94, 129, 172}
\definecolor{boxFill}{RGB}{236, 240, 241}
\definecolor{solLine}{RGB}{163, 190, 140}
\definecolor{expandColor}{RGB}{191, 97, 106}

% --- TikZ styles (DFA / NFA / PDA) ---
\tikzset{
    dfa_state/.style={
        state,
        thick,
        draw=accentColor,
        fill=boxFill,
        text=primaryColor,
        minimum size=1.1cm
    },
    dfa_edge/.style={
        ->,
        >=stealth,
        thick,
        draw=primaryColor,
        auto
    },
    pda_node/.style={
        state,
        thick,
        draw=accentColor,
        fill=boxFill,
        text=primaryColor,
        minimum size=1.2cm
    },
    pda_edge/.style={
        ->,
        >=stealth,
        thick,
        draw=primaryColor,
        auto
    },
    rule_expand/.style={
        pda_edge,
        draw=expandColor,
        text=expandColor
    },
    rule_match/.style={
        pda_edge,
        draw=solLine,
        text=solLine!80!black
    }
}

% --- Header / footer ---
\pagestyle{fancy}
\fancyhf{}
\lhead{\color{accentColor}\textbf{形式语言与自动机作业}}
\rhead{\color{primaryColor}\textbf{\HomeworkHeader}}
\cfoot{\small 第 \thepage \ 页,共 \pageref{LastPage} 页}
\setlength{\headheight}{24pt}
\addtolength{\topmargin}{-12pt}
\setlength{\emergencystretch}{2em}
\renewcommand{\headrulewidth}{0.4pt}
\renewcommand{\headrule}{\hbox to\headwidth{\color{accentColor}\leaders\hrule height \headrulewidth\hfill}}

% --- Problem box ---
\newtcolorbox[auto counter]{problem}[1][]{
    enhanced,
    breakable,
    colback=boxFill,
    colframe=accentColor,
    coltitle=white,
    fonttitle=\bfseries\large,
    title={题目 ~\thetcbcounter \quad #1},
    boxrule=0.5mm,
    arc=3mm,
    drop shadow=black!15!white,
    attach boxed title to top left={xshift=0.5cm, yshift*=-3mm},
    boxed title style={colback=accentColor}
}

% --- Solution environment ---
\newenvironment{solution}{
    \par\vspace{10pt}
    \noindent\textbf{\color{solLine} \Large \textit{Solution.}} \par
    \begingroup\color{primaryColor}\sloppy
    \leftskip1em \rightskip1em
    \noindent\rule{\textwidth}{0.4pt}
    \par\vspace{5pt}
}{
    \par\vspace{10pt}
    \noindent\rule[0.5ex]{\textwidth}{0.1pt}
    \endgroup
    \vspace{1.2cm}
}

% --- Title ---
\newcommand{\makeCustomTitle}[1]{
    \begin{center}
        \vspace*{1cm}
        {\Huge \bfseries \color{primaryColor} 作业 #1}\\
        \vspace{0.5cm}
        {\Large \color{accentColor} \today}
        \vspace{1.2cm}
    \end{center}
}


\begin{document}

\makeCustomTitle{L1.6 - L1.9:NFA 构造、正则表达式与正则变换}

\section*{说明(按题面顺序)}
题面 \texttt{assets/problem\_L01\_06-L01\_09.pdf} 中存在\textbf{重复题目}:
\begin{itemize}
  \item 第 2 页的“题 1”与第 1 页的“题 1/题 2”部分重复
\end{itemize}

\section*{字母表约定}
除非题目另行说明,本作业中所有 NFA/正则表达式的字母表均为 $\Sigma=\{0,1\}$。

% =========================================================
% Page 1
% =========================================================

\begin{problem}
(PDF 第 1 页·题 1)构造识别如下语言的 NFA,且符合规定的状态数:
\begin{enumerate}[label=(\alph*)]
  \item $\{w\mid w\text{ 以 }00\text{ 结束}\}$,3 个状态
  \item $\{w\mid w\text{ 含有偶数个 }0\ \text{或恰好 2 个 }1\}$,6 个状态
\end{enumerate}
\end{problem}

\begin{solution}
\textbf{(a) 以 00 结束(3 状态)}

取状态集合 $\{q_0,q_1,q_2\}$,初态 $q_0$,终态 $\{q_2\}$。定义转移:
\[
\begin{aligned}
\delta(q_0,0)&=\{q_0,q_1\}, & \delta(q_0,1)&=\{q_0\},\\
\delta(q_1,0)&=\{q_2\}, & \delta(q_1,1)&=\emptyset,\\
\delta(q_2,0)&=\{q_2\}, & \delta(q_2,1)&=\emptyset.
\end{aligned}
\]
直观:$q_0$ 在任意前缀上自环;当读到某个 $0$ 时非确定性“猜测它是倒数第二个 0”,转入 $q_1$,再读到 $0$ 到达接受态 $q_2$。

\textbf{(b) 偶数个 0 或恰好 2 个 1(6 状态)}

分别构造两个自动机并用 $\epsilon$-并联实现并集。

\emph{分支 1:偶数个 0(2 状态 DFA 亦可视作 NFA)}
状态 $\{e,o\}$:$e$ 表示已读到偶数个 0,$o$ 表示奇数个 0;初态 $e$,终态 $\{e\}$。
\[
\delta(e,0)=o,\ \delta(o,0)=e,\qquad
\delta(e,1)=e,\ \delta(o,1)=o.
\]

\emph{分支 2:恰好 2 个 1(3 状态 NFA)}
状态 $\{p_0,p_1,p_2\}$ 表示已读到的 $1$ 的个数为 $0/1/2$;初态 $p_0$,终态 $\{p_2\}$,读到第 3 个 $1$ 无路可走:
\[
\begin{aligned}
\delta(p_0,0)&=\{p_0\}, & \delta(p_0,1)&=\{p_1\},\\
\delta(p_1,0)&=\{p_1\}, & \delta(p_1,1)&=\{p_2\},\\
\delta(p_2,0)&=\{p_2\}, & \delta(p_2,1)&=\emptyset.
\end{aligned}
\]

\emph{并联合成(总 6 状态)}
新增总初态 $s$:
\[
\delta(s,\epsilon)=\{e,p_0\},
\]
总终态为 $\{e,p_2\}$。总状态数 $1+2+3=6$,满足题目要求。
\end{solution}

\begin{problem}
(PDF 第 1 页·题 2)构造识别如下语言的 NFA:
\[
A=\{w\mid w\text{ 含有子串 }0101\},\qquad
B=\{w\mid w\text{ 不含子串 }110\}.
\]
并构造识别 $A\cup B$、$A\circ B$(连接)与 $B^*$ 的 NFA。
\end{problem}

\begin{solution}
\textbf{(1) $A$:含有子串 0101(5 状态)}

用“匹配到前缀长度”的 NFA($a_0$ 初态,$a_4$ 终态):
\[
\begin{aligned}
\delta(a_0,0)&=\{a_0,a_1\}, & \delta(a_0,1)&=\{a_0\},\\
\delta(a_1,1)&=\{a_2\}, & \delta(a_1,0)&=\emptyset,\\
\delta(a_2,0)&=\{a_3\}, & \delta(a_2,1)&=\emptyset,\\
\delta(a_3,1)&=\{a_4\}, & \delta(a_3,0)&=\emptyset,\\
\delta(a_4,0)&=\{a_4\}, & \delta(a_4,1)&=\{a_4\}.
\end{aligned}
\]
其中 $a_0$ 的自环实现“任意前缀”,并在读到某个 $0$ 时非确定性开始匹配。

\textbf{(2) $B$:不含子串 110(DFA 亦可视作 NFA)}

状态 $\{b_0,b_1,b_2,\bot\}$,含义分别为“末尾不是 1”“末尾是 1”“末尾是 11”“已出现 110(陷阱)”。
初态 $b_0$,终态 $\{b_0,b_1,b_2\}$:
\[
\begin{aligned}
\delta(b_0,0)&=b_0, & \delta(b_0,1)&=b_1,\\
\delta(b_1,0)&=b_0, & \delta(b_1,1)&=b_2,\\
\delta(b_2,1)&=b_2, & \delta(b_2,0)&=\bot,\\
\delta(\bot,0)&=\bot, & \delta(\bot,1)&=\bot.
\end{aligned}
\]

\textbf{(3) $A\cup B$(并集)}

新增初态 $s$,$\delta(s,\epsilon)=\{a_0,b_0\}$,终态取 $F_A\cup F_B$ 即可。

\textbf{(4) $A\circ B$(连接)}

把 $A$ 的接受态 $a_4$ 用 $\epsilon$ 边连到 $B$ 的初态 $b_0$,并把整体接受态设为 $B$ 的接受态 $\{b_0,b_1,b_2\}$。

\textbf{(5) $B^*$(星号)}

新增接受初态 $s$(用于接受 $\epsilon$),$\delta(s,\epsilon)=\{b_0\}$,
并对每个 $B$ 的接受态 $f\in\{b_0,b_1,b_2\}$ 加回边 $\delta(f,\epsilon)\ni b_0$。
\end{solution}

% =========================================================
% Page 2 (duplicate page)
% =========================================================

\begin{problem}
(PDF 第 2 页·题 1)给出识别下述语言的 NFA,且符合规定的状态数(题面与前面有重复):
\begin{enumerate}[label=(\alph*)]
  \item $\{w\mid w\text{ 以 }00\text{ 结束}\}$,3 个状态(\textbf{与第 1 页·题 1(a) 重复})
  \item $0^*1^*0^+$,3 个状态(\textbf{新增})
  \item $\{w\mid w\text{ 含有子串 }0101\}$,5 个状态(\textbf{与第 1 页·题 2 的 $A$ 重复})
  \item $\{w\mid w\text{ 含有偶数个 }0\ \text{或恰好 2 个 }1\}$,6 个状态(\textbf{与第 1 页·题 1(b) 重复})
\end{enumerate}
\end{problem}

\begin{solution}
\textbf{(a) 重复说明:}同第 1 页·题 1(a)。

\textbf{(c) 重复说明:}同第 1 页·题 2 中对 $A$ 的 5 状态构造。

\textbf{(d) 重复说明:}同第 1 页·题 1(b) 的 6 状态并联构造。

\textbf{(b) $0^*1^*0^+$(3 状态)}

该语言表示:先若干个 0,再若干个 1,最后至少一个 0(且一旦进入 1 段后,只能在末尾进入 0 段并停留在 0 段)。
给出 3 状态 DFA(也是 NFA):

状态 $\{q_0,q_1,q_2\}$,初态 $q_0$,终态 $\{q_2\}$:
\[
\begin{aligned}
\delta(q_0,0)&=q_0, & \delta(q_0,1)&=q_1,\\
\delta(q_1,1)&=q_1, & \delta(q_1,0)&=q_2,\\
\delta(q_2,0)&=q_2, & \delta(q_2,1)&=\emptyset.
\end{aligned}
\]
其中 $q_2$ 表示已经进入末尾的 $0^+$ 段;若再读到 1 则不可能回到 $1^*$,因此直接拒绝(无转移)。
\end{solution}

% =========================================================
% Page 3
% =========================================================

\begin{problem}
(PDF 第 3 页·题 2)给出识别下述语言\textbf{并集}的 NFA 状态图:
\[
L=\{w\mid w\text{ 从 }1\text{ 开始且以 }0\text{ 结束}\}\ \cup\ \{w\mid w\text{ 含有至少 3 个 }1\}.
\]
\end{problem}

\begin{solution}
分别为两个语言构造 DFA/NFA,再用 $\epsilon$-并联实现并集(与前面 $A\cup B$ 的做法相同)。

\textbf{语言 1:从 1 开始且以 0 结束}

状态 $\{s,\alpha,\beta,\bot\}$,$s$ 初态,$\beta$ 终态:
\[
\begin{aligned}
\delta(s,1)&=\alpha, & \delta(s,0)&=\bot,\\
\delta(\alpha,1)&=\alpha, & \delta(\alpha,0)&=\beta,\\
\delta(\beta,0)&=\beta, & \delta(\beta,1)&=\alpha,\\
\delta(\bot,0)&=\bot, & \delta(\bot,1)&=\bot.
\end{aligned}
\]

\textbf{语言 2:至少 3 个 1}

状态 $\{c_0,c_1,c_2,c_3\}$,$c_0$ 初态,$c_3$ 终态:
\[
\delta(c_i,0)=c_i,\quad
\delta(c_0,1)=c_1,\ \delta(c_1,1)=c_2,\ \delta(c_2,1)=c_3,\ \delta(c_3,1)=c_3.
\]

\textbf{并联:}新增初态 $s_*$,$\delta(s_*,\epsilon)=\{s,c_0\}$,终态取 $\{\beta,c_3\}$。
\end{solution}

\begin{problem}
(PDF 第 3 页·题 3)给出识别下述语言\textbf{连接}的 NFA 状态图:
\[
L=\{w\mid |w|\le 5\}\ \circ\ \{w\mid w\text{ 的奇数位置均为 }1\}.
\]
\end{problem}

\begin{solution}
\textbf{语言 1:$|w|\le 5$}

用 6 个状态计长度:$l_0$(初态)到 $l_5$ 全为终态;读入任意符号推进;超过 5 无路:
\[
\delta(l_i,0)=l_{i+1},\ \delta(l_i,1)=l_{i+1}\quad (0\le i\le 4),\qquad
\delta(l_5,0)=\delta(l_5,1)=\emptyset.
\]

\textbf{语言 2:奇数位置均为 1}

状态 $\{O,E,\bot\}$:$O$ 表示“下一位是奇数位”,$E$ 表示“下一位是偶数位”,$\bot$ 陷阱。
初态 $O$,终态 $\{O,E\}$:
\[
\delta(O,1)=E,\ \delta(O,0)=\bot,\qquad
\delta(E,0)=O,\ \delta(E,1)=O,\qquad
\delta(\bot,0)=\delta(\bot,1)=\bot.
\]

\textbf{连接:}把长度机的每个终态 $l_i$($0\le i\le 5$)用 $\epsilon$ 边连到 $O$,整体终态为 $\{O,E\}$。
\end{solution}

\begin{problem}
(PDF 第 3 页·题 4)给出识别下述语言\textbf{星号}的 NFA 状态图:
\begin{enumerate}[label=(\alph*)]
  \item $\{w\mid w\text{ 含有至少 3 个 }1\}$
  \item $\{w\mid w\text{ 含有至少 2 个 }0\text{ 且至多含有 1 个 }1\}$
\end{enumerate}
\end{problem}

\begin{solution}
做法:先给出识别 $L$ 的 NFA(或 DFA),再按 Kleene 星标准构造:新增接受初态 $s$,$\delta(s,\epsilon)=\{\text{原初态}\}$,并从原每个接受态加 $\epsilon$ 回到原初态。

\textbf{(a) 至少 3 个 1 的语言 $L$}

可直接复用上题“至少 3 个 1”的 DFA($c_0,c_1,c_2,c_3$),将其按上述方式变为 $L^*$。

\textbf{(b) 至少 2 个 0 且至多 1 个 1 的语言 $K$}

用 DFA 记录 $(z,o)$:$z\in\{0,1,2\}$ 表示 0 的个数(2 表示 $\ge 2$),$o\in\{0,1,2\}$ 表示 1 的个数(2 表示 $\ge 2$,为陷阱)。
初态 $(0,0)$,终态为 $\{(2,0),(2,1)\}$,转移(饱和计数):
\[
\delta((z,o),0)=(\min(z+1,2),o),\qquad
\delta((z,o),1)=(z,\min(o+1,2)).
\]
再按 Kleene 星构造得到 $K^*$。
\end{solution}

% =========================================================
% Page 4
% =========================================================

\begin{problem}
(PDF 第 4 页·题 1)写出表示下列语言的正则表达式:
\begin{enumerate}[label=(\arabic*)]
  \item 字母表 $\{a,b,c\}$ 上包含至少一个 $a$ 和至少一个 $b$ 的串
  \item 倒数第 10 个符号是 1 的 0/1 串
  \item 至多只有一对连续 1 的 0/1 串
  \item 0 的个数被 5 整除的 0/1 串
  \item 不包括 101 作为子串的所有 0/1 串
  \item 0 的个数被 5 整除且 1 的个数是偶数的所有 0/1 串
\end{enumerate}
\end{problem}

\begin{solution}
\textbf{(1) 至少一个 $a$ 且至少一个 $b$($\Sigma=\{a,b,c\}$)}

令 $T=(a+b+c)$,则:
\[
T^*aT^*bT^* + T^*bT^*aT^*.
\]

\textbf{(2) 倒数第 10 个符号是 1($\Sigma=\{0,1\}$)}
\[
(0+1)^*\,1\,(0+1)^9.
\]

\textbf{(3) 至多只有一对连续 1}

先记“不含 11”的语言为 $N=(0+10)^*(\epsilon+1)$。
“恰好一次出现 11”可写为:前缀必须以 0 或空结尾(避免形成 111),后缀必须以 0 或空开头(避免形成 111):
\[
(0+10)^*\,11\,(\epsilon+0(0+10)^*(\epsilon+1)).
\]
所以答案为
\[
N + (0+10)^*\,11\,(\epsilon+0(0+10)^*(\epsilon+1)).
\]

\textbf{(4) 0 的个数被 5 整除}
\[
(1^*01^*01^*01^*01^*0)^*1^*.
\]

\textbf{(5) 不含子串 101}

等价表述:所有的 $1$-块之间至少隔两个 0(否则会出现 $\cdots 1\,0\,1\cdots$)。因此:
\[
0^*\big(\epsilon + 1^+(00\,0^*\,1^+)^*0^*\big).
\]

\textbf{(6) 0 的个数被 5 整除且 1 的个数为偶数}

可构造一个 DFA 以记录“0 的个数 mod 5”与“1 的个数 mod 2”的直积状态(共 $10$ 状态),因此该语言正则。
用状态消除法可得到一个等价正则表达式,例如:
\[
\begin{aligned}
R_6
=\ &\Big(11+((01+10)(11)^*00+(00+(01+10)(11)^*(10+01))(11)^*(10+01))\\
&\qquad\cdot(11+0(00(11)^*00+00(11)^*(10+01)(11)^*(10+01)))^*001\\
&\qquad+((00+(01+10)(11)^*(10+01))(11)^*00\\
&\qquad\quad+((01+10)(11)^*00+(00+(01+10)(11)^*(10+01))(11)^*(10+01))\\
&\qquad\quad\cdot(11+0(00(11)^*00+00(11)^*(10+01)(11)^*(10+01)))^*\\
&\qquad\quad\cdot(10+0(1+00(11)^*(10+01)(11)^*00))\Big)\\
&\qquad\cdot\Big(1(1+00(11)^*(10+01)(11)^*00)\\
&\qquad\quad+1(00(11)^*00+00(11)^*(10+01)(11)^*(10+01))\\
&\qquad\quad\cdot(11+0(00(11)^*00+00(11)^*(10+01)(11)^*(10+01)))^*\\
&\qquad\quad\cdot(10+0(1+00(11)^*(10+01)(11)^*00))\Big)^*\\
&\qquad\cdot(0+101+1(00(11)^*00+00(11)^*(10+01)(11)^*(10+01))\\
&\qquad\quad\cdot(11+0(00(11)^*00+00(11)^*(10+01)(11)^*(10+01)))^*001)\Big)^*.
\end{aligned}
\]
(该表达式较长,但与题目语言等价,且满足“正则表达式”形式要求。)
\end{solution}

% =========================================================
% Page 5
% =========================================================

\begin{problem}
(PDF 第 5 页·题 2)给出下列正则表达式语言的自然语言描述:
\begin{enumerate}[label=(\arabic*)]
  \item $(1+\epsilon)(00^*1)^*0^*$
  \item $(0^*1^*)^*000(0+1)^*$
  \item $(0+10)^*1^*$
\end{enumerate}
\end{problem}

\begin{solution}
\textbf{(1) $(1+\epsilon)(00^*1)^*0^*$}

可选地以一个 $1$ 开头(也可以不选),然后重复若干次“若干个 $0$(至少 1 个)再跟一个 $1$”的块,最后以若干个 $0$ 结束。
因此该语言中的串满足:\emph{除可能存在的首个 $1$ 外,每个 $1$ 的前面至少有一个 $0$;并且串以一段 $0$ 结束。}

\textbf{(2) $(0^*1^*)^*000(0+1)^*$}

因为 $(0^*1^*)^*=\Sigma^*$(任意 0/1 串都可分割成若干段“若干 0 后接若干 1”),所以该表达式等价于
\[
\Sigma^*000\Sigma^*,
\]
即:\emph{包含子串 \texttt{000} 的所有 0/1 串。}

\textbf{(3) $(0+10)^*1^*$}

前半部分 $(0+10)^*$ 表示可由若干个 \texttt{0} 或 \texttt{10} 拼接得到(因此该部分若出现 1,则该 1 必紧跟一个 0),再接一个末尾的 $1^*$。
等价描述:\emph{除末尾可能出现的一段连续 1 外,其余位置的每个 1 都必须紧跟一个 0。}
\end{solution}

% =========================================================
% Page 6
% =========================================================

\begin{problem}
(PDF 第 6 页)把下列正则表达式转换成带 $\epsilon$ 转移的 NFA:
\[
1)\ 01^*,\qquad
2)\ (0+1)01,\qquad
3)\ 00(0+1)^*.
\]
\end{problem}

\begin{solution}
用 Thompson 构造法即可。下面用“状态 + 边”的方式给出 $\epsilon$-NFA(也可直接画成状态图)。

\textbf{(1) $01^*$}

状态 $\{s,u,v,f\}$,初态 $s$,终态 $f$:
\[
s\xrightarrow{0}u,\quad
u\xrightarrow{\epsilon}f,\quad
u\xrightarrow{1}v,\quad
v\xrightarrow{\epsilon}u.
\]
其中 $u\to f$ 允许取 $1^*$ 的 0 次,$u\xrightarrow{1}v\xrightarrow{\epsilon}u$ 实现循环。

\textbf{(2) $(0+1)01$}

先做并:$0+1$,再连接 $01$。
状态 $\{s,a,b,c,d,f\}$,初态 $s$,终态 $f$:
\[
\begin{aligned}
s&\xrightarrow{\epsilon}a,\ s\xrightarrow{\epsilon}b,\\
a&\xrightarrow{0}c,\quad b\xrightarrow{1}c,\\
c&\xrightarrow{0}d,\quad d\xrightarrow{1}f.
\end{aligned}
\]

\textbf{(3) $00(0+1)^*$}

先用两条边读入前缀 \texttt{00},再对 $(0+1)^*$ 加星号。
状态 $\{s,u,v,x,y,f\}$,初态 $s$,终态 $f$:
\[
\begin{aligned}
s&\xrightarrow{0}u,\quad u\xrightarrow{0}v,\\
v&\xrightarrow{\epsilon}f,\quad v\xrightarrow{\epsilon}x,\\
x&\xrightarrow{0}y,\ x\xrightarrow{1}y,\quad y\xrightarrow{\epsilon}x,\ y\xrightarrow{\epsilon}f.
\end{aligned}
\]
其中 $v\to f$ 对应星号取 0 次;$x\xrightarrow{0/1}y$ 再回到 $x$ 实现任意多次。
\end{solution}

% =========================================================
% Page 7
% =========================================================

\begin{problem}
(PDF 第 7 页)利用本节课讲授的方法,把题面给出的 DFA 转换成正则表达式(两小题,见图 (a)(b))。
\end{problem}

\begin{solution}
\textbf{(a) 两状态 DFA:$a$ 自环,$b$ 在两状态间切换,且仅状态 2 为终态}

该 DFA 接受“$b$ 的出现次数为奇数”的语言($a$ 不改变状态,$b$ 翻转状态)。
正则表达式可写为:
\[
\boxed{a^*(ba^*ba^*)^*ba^*}.
\]

\textbf{(b) 三状态 DFA:初态 1 与状态 3 为终态(见题图)}

设 $K=(a+b)(a+bb)^*ba$。用 Arden 引理/状态消除法可得从初态到终态的语言为:
\[
\boxed{K^*\big(\epsilon+(a+b)(a+bb)^*b\big)}.
\]
\end{solution}

\end{document}

