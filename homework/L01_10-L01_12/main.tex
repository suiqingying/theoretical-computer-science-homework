\documentclass[11pt, a4paper]{article}
\newcommand{\HomeworkHeader}{L01\_10-L01\_12}
% Shared LaTeX preamble for homework/*/main.tex

% --- Base layout ---
\usepackage[a4paper, top=2.5cm, bottom=2.5cm, left=2cm, right=2cm]{geometry}
\usepackage{fontspec}
\usepackage{xeCJK} % CJK fallback to avoid missing glyphs in Latin fonts
\usepackage{amsmath, amssymb, amsthm}
\usepackage{mathtools}
\usepackage{fancyhdr}
\usepackage{lastpage}
\usepackage[svgnames]{xcolor}
\usepackage{tikz}
\usetikzlibrary{automata, positioning, arrows.meta, bending, backgrounds}
\usepackage{pgfplots}
\pgfplotsset{compat=1.18}
\usepackage[most]{tcolorbox}
\usepackage{enumitem}

% --- Languages & fonts ---
\usepackage[english]{babel}
\babelprovide[import, main]{chinese}

\babelfont{rm}[UprightFont=*, BoldFont=* Bold, ItalicFont=* Italic, Scale=1.05]{Linux Libertine O}
\babelfont{sf}[UprightFont=*, BoldFont=* Bold, Scale=1.05]{Linux Biolinum O}
\babelfont[chinese]{rm}[
    UprightFont=*-Regular,
    BoldFont=*-Bold,
    ItalicFont=*-Regular,
    BoldItalicFont=*-Bold
]{Noto Serif CJK SC}
\babelfont[chinese]{sf}[
    UprightFont=*-Regular,
    BoldFont=*-Bold
]{Noto Sans CJK SC}
\babelfont{tt}{Noto Sans Mono}
\babelfont[chinese]{tt}{Noto Sans Mono CJK SC}

% xeCJK font setup (kept consistent with babelfont)
\setCJKmainfont{Noto Serif CJK SC}
\setCJKsansfont{Noto Sans CJK SC}
\setCJKmonofont{Noto Sans Mono CJK SC}

% --- Colors ---
\definecolor{primaryColor}{RGB}{46, 52, 64}
\definecolor{accentColor}{RGB}{94, 129, 172}
\definecolor{boxFill}{RGB}{236, 240, 241}
\definecolor{solLine}{RGB}{163, 190, 140}
\definecolor{expandColor}{RGB}{191, 97, 106}

% --- TikZ styles (DFA / NFA / PDA) ---
\tikzset{
    dfa_state/.style={
        state,
        thick,
        draw=accentColor,
        fill=boxFill,
        text=primaryColor,
        minimum size=1.1cm
    },
    dfa_edge/.style={
        ->,
        >=stealth,
        thick,
        draw=primaryColor,
        auto
    },
    pda_node/.style={
        state,
        thick,
        draw=accentColor,
        fill=boxFill,
        text=primaryColor,
        minimum size=1.2cm
    },
    pda_edge/.style={
        ->,
        >=stealth,
        thick,
        draw=primaryColor,
        auto
    },
    rule_expand/.style={
        pda_edge,
        draw=expandColor,
        text=expandColor
    },
    rule_match/.style={
        pda_edge,
        draw=solLine,
        text=solLine!80!black
    }
}

% --- Header / footer ---
\pagestyle{fancy}
\fancyhf{}
\lhead{\color{accentColor}\textbf{形式语言与自动机作业}}
\rhead{\color{primaryColor}\textbf{\HomeworkHeader}}
\cfoot{\small 第 \thepage \ 页,共 \pageref{LastPage} 页}
\setlength{\headheight}{24pt}
\addtolength{\topmargin}{-12pt}
\setlength{\emergencystretch}{2em}
\renewcommand{\headrulewidth}{0.4pt}
\renewcommand{\headrule}{\hbox to\headwidth{\color{accentColor}\leaders\hrule height \headrulewidth\hfill}}

% --- Problem box ---
\newtcolorbox[auto counter]{problem}[1][]{
    enhanced,
    breakable,
    colback=boxFill,
    colframe=accentColor,
    coltitle=white,
    fonttitle=\bfseries\large,
    title={题目 ~\thetcbcounter \quad #1},
    boxrule=0.5mm,
    arc=3mm,
    drop shadow=black!15!white,
    attach boxed title to top left={xshift=0.5cm, yshift*=-3mm},
    boxed title style={colback=accentColor}
}

% --- Solution environment ---
\newenvironment{solution}{
    \par\vspace{10pt}
    \noindent\textbf{\color{solLine} \Large \textit{Solution.}} \par
    \begingroup\color{primaryColor}\sloppy
    \leftskip1em \rightskip1em
    \noindent\rule{\textwidth}{0.4pt}
    \par\vspace{5pt}
}{
    \par\vspace{10pt}
    \noindent\rule[0.5ex]{\textwidth}{0.1pt}
    \endgroup
    \vspace{1.2cm}
}

% --- Title ---
\newcommand{\makeCustomTitle}[1]{
    \begin{center}
        \vspace*{1cm}
        {\Huge \bfseries \color{primaryColor} 作业 #1}\\
        \vspace{0.5cm}
        {\Large \color{accentColor} \today}
        \vspace{1.2cm}
    \end{center}
}


\begin{document}

\makeCustomTitle{L1.10 - L1.12:正则表达式与 Arden 引理}

% --- Problem 1 ---
\begin{problem}
用分配律化简如下正则表达式,得到两个不同但更简单的等价表达式:
\[
(0+1)^*1(0+1)(0+1) + (0+1)^*1(0+1).
\]
\end{problem}

\begin{solution}
设 $X=(0+1)$,原式为 $X^*1XX + X^*1X$。

\textbf{化简 1:右因子提取}
\[
X^*1XX + X^*1X = X^*1X(X+\epsilon).
\]
其中 $(X+\epsilon)$ 表示“可选的一个 $X$”,因此该式更简洁地表达为“在某个 1 之后跟 1 个或 2 个符号”。

\textbf{化简 2:左因子提取}
\[
X^*1XX + X^*1X = X^*1(XX+X)=X^*1X(X+\epsilon).
\]
两种写法分别是提取右因子/左因子,均比原式更紧凑,且等价。
\end{solution}

% --- Problem 2 ---
\begin{problem}
证明:$(L+M)^* = (L^*M^*)^*$。
\end{problem}

\begin{solution}
证明两边语言互相包含。

\textbf{(1) $(L+M)^* \subseteq (L^*M^*)^*$}

任取 $w\in (L+M)^*$,则 $w$ 可写为有限个因子连接:
\[
w = x_1x_2\cdots x_n,\quad x_i\in L\ \text{或}\ x_i\in M.
\]
把相邻的 $L$-因子合并成一个 $L^*$-块,相邻的 $M$-因子合并成一个 $M^*$-块,可得
\[
w = y_1y_2\cdots y_k,
\]
其中每个 $y_j$ 属于 $L^*$ 或 $M^*$,并且它们在序列中交替出现(可能从 $L^*$ 或 $M^*$ 开始)。
将相邻的一对 $(L^*)(M^*)$ 视为一个块(允许其中一个为空串 $\epsilon$),则每个块属于 $L^*M^*$,因此 $w\in (L^*M^*)^*$。

\textbf{(2) $(L^*M^*)^* \subseteq (L+M)^*$}

任取 $w\in (L^*M^*)^*$,则
\[
w = z_1z_2\cdots z_t,\quad z_i\in L^*M^*.
\]
每个 $z_i$ 可写为 $z_i=uv$,其中 $u\in L^*$,$v\in M^*$。而 $u$ 是若干个 $L$ 中串的连接,$v$ 是若干个 $M$ 中串的连接,
因此 $z_i$ 也是由若干个属于 $L$ 或 $M$ 的串连接得到,所以 $z_i\in (L+M)^*$。于是 $w$ 作为若干个 $z_i$ 的连接仍属于 $(L+M)^*$。

综上,两边相等:$(L+M)^* = (L^*M^*)^*$。
\end{solution}

\section*{L1.12:利用 Arden 引理求正则表达式}

% --- Problem 3 ---
\begin{problem}
利用 Arden 引理将如下有穷自动机转换成正则表达式。
\begin{enumerate}[label=(\arabic*)]
    \item 三状态自动机:起始态 $A$,接受态 $C$。转移为
    \[
    A \xrightarrow{a} A,\quad A \xrightarrow{a} B,\quad B \xrightarrow{b} B,\quad B \xrightarrow{b} A,\quad
    B \xrightarrow{a} C,\quad C \xrightarrow{b} B.
    \]
    \item 三状态自动机:起始态 $q_1$,接受态 $q_3$。转移为
    \[
    q_1 \xrightarrow{a} q_1,\quad q_1 \xrightarrow{b} q_2,\quad q_2 \xrightarrow{b} q_2,\quad
    q_2 \xrightarrow{a} q_3,\quad q_2 \xrightarrow{a} q_1,\quad q_3 \xrightarrow{a} q_2.
    \]
\end{enumerate}
\end{problem}

\begin{solution}
对每个状态 $X$,令 $R_X$ 表示“从起始态出发到达 $X$ 的所有串”所构成的正则表达式(语言)。

\textbf{(1) 自动机 (A,B,C)}

由转移关系写方程:
\[
\begin{aligned}
R_A &= \epsilon + R_A a + R_B b,\\
R_B &= R_A a + R_B b + R_C b,\\
R_C &= R_B a.
\end{aligned}
\]
先由第三式代入第二式:
\[
R_B = R_A a + R_B b + (R_B a)b = R_A a + R_B(b+ab).
\]
由 Arden 引理(若 $X=XA+Y$ 且 $\epsilon\notin A$,则 $X=YA^*$)得
\[
R_B = R_A a (b+ab)^*.
\]
代回第一式:
\[
R_A = \epsilon + R_A a + \big(R_A a (b+ab)^*\big)b
     = \epsilon + R_A\big(a + a(b+ab)^*b\big).
\]
再次用 Arden 引理:
\[
R_A = \epsilon \big(a + a(b+ab)^*b\big)^*.
\]
因此
\[
R_C = R_B a = R_A a (b+ab)^* a
= \big(a + a(b+ab)^*b\big)^*\, a(b+ab)^* a.
\]
接受态为 $C$,所以所求正则表达式可取:
\[
\boxed{ \big(a + a(b+ab)^*b\big)^*\, a(b+ab)^* a }.
\]

\textbf{(2) 自动机 $(q_1,q_2,q_3)$}

写方程:
\[
\begin{aligned}
R_1 &= \epsilon + R_1 a + R_2 a,\\
R_2 &= R_1 b + R_2 b + R_3 a,\\
R_3 &= R_2 a.
\end{aligned}
\]
代入 $R_3=R_2 a$ 到第二式:
\[
R_2 = R_1 b + R_2 b + (R_2 a)a = R_1 b + R_2(b+aa).
\]
由 Arden 引理:
\[
R_2 = R_1 b (b+aa)^*.
\]
代回第一式:
\[
R_1 = \epsilon + R_1 a + \big(R_1 b (b+aa)^*\big)a
= \epsilon + R_1\big(a + b(b+aa)^*a\big),
\]
因此
\[
R_1 = \big(a + b(b+aa)^*a\big)^*.
\]
接受态为 $q_3$,故
\[
R_3 = R_2 a = R_1 b (b+aa)^* a
= \big(a + b(b+aa)^*a\big)^*\, b(b+aa)^* a.
\]
所以可取正则表达式:
\[
\boxed{ \big(a + b(b+aa)^*a\big)^*\, b(b+aa)^* a }.
\]
\end{solution}

\end{document}
