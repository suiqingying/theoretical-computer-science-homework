\documentclass[11pt, a4paper]{article}
\newcommand{\HomeworkHeader}{L02\_14-L03\_02}
% Shared LaTeX preamble for homework/*/main.tex

% --- Base layout ---
\usepackage[a4paper, top=2.5cm, bottom=2.5cm, left=2cm, right=2cm]{geometry}
\usepackage{fontspec}
\usepackage{xeCJK} % CJK fallback to avoid missing glyphs in Latin fonts
\usepackage{amsmath, amssymb, amsthm}
\usepackage{mathtools}
\usepackage{fancyhdr}
\usepackage{lastpage}
\usepackage[svgnames]{xcolor}
\usepackage{tikz}
\usetikzlibrary{automata, positioning, arrows.meta, bending, backgrounds}
\usepackage{pgfplots}
\pgfplotsset{compat=1.18}
\usepackage[most]{tcolorbox}
\usepackage{enumitem}

% --- Languages & fonts ---
\usepackage[english]{babel}
\babelprovide[import, main]{chinese}

\babelfont{rm}[UprightFont=*, BoldFont=* Bold, ItalicFont=* Italic, Scale=1.05]{Linux Libertine O}
\babelfont{sf}[UprightFont=*, BoldFont=* Bold, Scale=1.05]{Linux Biolinum O}
\babelfont[chinese]{rm}[
    UprightFont=*-Regular,
    BoldFont=*-Bold,
    ItalicFont=*-Regular,
    BoldItalicFont=*-Bold
]{Noto Serif CJK SC}
\babelfont[chinese]{sf}[
    UprightFont=*-Regular,
    BoldFont=*-Bold
]{Noto Sans CJK SC}
\babelfont{tt}{Noto Sans Mono}
\babelfont[chinese]{tt}{Noto Sans Mono CJK SC}

% xeCJK font setup (kept consistent with babelfont)
\setCJKmainfont{Noto Serif CJK SC}
\setCJKsansfont{Noto Sans CJK SC}
\setCJKmonofont{Noto Sans Mono CJK SC}

% --- Colors ---
\definecolor{primaryColor}{RGB}{46, 52, 64}
\definecolor{accentColor}{RGB}{94, 129, 172}
\definecolor{boxFill}{RGB}{236, 240, 241}
\definecolor{solLine}{RGB}{163, 190, 140}
\definecolor{expandColor}{RGB}{191, 97, 106}

% --- TikZ styles (DFA / NFA / PDA) ---
\tikzset{
    dfa_state/.style={
        state,
        thick,
        draw=accentColor,
        fill=boxFill,
        text=primaryColor,
        minimum size=1.1cm
    },
    dfa_edge/.style={
        ->,
        >=stealth,
        thick,
        draw=primaryColor,
        auto
    },
    pda_node/.style={
        state,
        thick,
        draw=accentColor,
        fill=boxFill,
        text=primaryColor,
        minimum size=1.2cm
    },
    pda_edge/.style={
        ->,
        >=stealth,
        thick,
        draw=primaryColor,
        auto
    },
    rule_expand/.style={
        pda_edge,
        draw=expandColor,
        text=expandColor
    },
    rule_match/.style={
        pda_edge,
        draw=solLine,
        text=solLine!80!black
    }
}

% --- Header / footer ---
\pagestyle{fancy}
\fancyhf{}
\lhead{\color{accentColor}\textbf{形式语言与自动机作业}}
\rhead{\color{primaryColor}\textbf{\HomeworkHeader}}
\cfoot{\small 第 \thepage \ 页,共 \pageref{LastPage} 页}
\setlength{\headheight}{24pt}
\addtolength{\topmargin}{-12pt}
\setlength{\emergencystretch}{2em}
\renewcommand{\headrulewidth}{0.4pt}
\renewcommand{\headrule}{\hbox to\headwidth{\color{accentColor}\leaders\hrule height \headrulewidth\hfill}}

% --- Problem box ---
\newtcolorbox[auto counter]{problem}[1][]{
    enhanced,
    breakable,
    colback=boxFill,
    colframe=accentColor,
    coltitle=white,
    fonttitle=\bfseries\large,
    title={题目 ~\thetcbcounter \quad #1},
    boxrule=0.5mm,
    arc=3mm,
    drop shadow=black!15!white,
    attach boxed title to top left={xshift=0.5cm, yshift*=-3mm},
    boxed title style={colback=accentColor}
}

% --- Solution environment ---
\newenvironment{solution}{
    \par\vspace{10pt}
    \noindent\textbf{\color{solLine} \Large \textit{Solution.}} \par
    \begingroup\color{primaryColor}\sloppy
    \leftskip1em \rightskip1em
    \noindent\rule{\textwidth}{0.4pt}
    \par\vspace{5pt}
}{
    \par\vspace{10pt}
    \noindent\rule[0.5ex]{\textwidth}{0.1pt}
    \endgroup
    \vspace{1.2cm}
}

% --- Title ---
\newcommand{\makeCustomTitle}[1]{
    \begin{center}
        \vspace*{1cm}
        {\Huge \bfseries \color{primaryColor} 作业 #1}\\
        \vspace{0.5cm}
        {\Large \color{accentColor} \today}
        \vspace{1.2cm}
    \end{center}
}


\begin{document}

\makeCustomTitle{L2.14 - L3.2:CFL 的性质与 CFL 泵引理}

% --- Problem 1 ---
\begin{problem}
证明下述语言是 CFL:
\[
L_1=\{a^n b^n \mid n\ge 0 \text{ 且 } n \text{ 不是 } 5 \text{ 的倍数}\}.
\]
\end{problem}

\begin{solution}
直接构造 CFG 生成该语言。注意 $n$ 不是 5 的倍数当且仅当
\[
n = 5k + r,\qquad k\ge 0,\ r\in\{1,2,3,4\}.
\]
构造文法:
\[
\boxed{
\begin{aligned}
S &\to S_1 \mid S_2 \mid S_3 \mid S_4,\\
S_1 &\to aUb,\\
S_2 &\to aaUbb,\\
S_3 &\to aaaUbbb,\\
S_4 &\to aaaaUbbbb,\\
U &\to aaaaaUbbbbb \mid \epsilon.
\end{aligned}}
\]
$U$ 生成 $a^{5k}b^{5k}$;$S_r$ 在两侧各补 $r$ 个符号,生成 $a^{5k+r}b^{5k+r}$($r=1,2,3,4$),正好对应 $n$ 不是 5 的倍数的所有情形。
因此 $L_1$ 为 CFL。
\end{solution}

% --- Problem 2 ---
\begin{problem}
证明下述语言是 CFL:
\[
L_2=\{w\in\{a,b\}^* \mid n_a(w)=n_b(w),\ w \text{ 不含子串 } \texttt{aab}\}.
\]
\end{problem}

\begin{solution}
设
\[
L_{bal}=\{w\in\{a,b\}^* \mid n_a(w)=n_b(w)\},
\qquad
R=\{w\in\{a,b\}^*\mid w\ \text{不含}\ \texttt{aab}\}.
\]
$L_{bal}$ 是 CFL:可由经典文法 $S\to aSbS\mid bSaS\mid \epsilon$ 生成所有 $a,b$ 数目相等的串。
$R$ 是正则语言(“不含固定子串”可由 DFA 识别)。
于是
\[
L_2 = L_{bal}\cap R
\]
为 CFL(CFL 与正则语言交封闭)。
\end{solution}

% --- Problem 3 ---
\begin{problem}
证明存在算法,可判定一个 CFL 是否包含长度小于 $n$ 的串。
\end{problem}

\begin{solution}
给定 CFL 的 CFG $G$ 与整数 $n$。
构造正则语言
\[
R_{<n}=\{w\mid |w|<n\}.
\]
$R_{<n}$ 可由一个 $n$ 个状态的 DFA 识别(逐字符计数到 $n$ 即转入拒绝陷阱)。

由于 CFL 与正则语言交封闭,$L(G)\cap R_{<n}$ 仍是 CFL。
并且 CFL 的空性问题可判定:对 CFG 可通过“可生成符号/可达符号”分析判断是否能生成某个终结串。

因此算法为:
\[
\text{判断 } L(G)\cap R_{<n} \stackrel{?}{=} \emptyset.
\]
若非空,则存在长度 $<n$ 的串;否则不存在。
\end{solution}

% --- Problem 4 ---
\begin{problem}
$L_1$ 为 CFL,$L_2$ 为正则语言,证明存在算法可判定 $L_1$ 和 $L_2$ 是否包含相同元素(即 $L_1\cap L_2\ne\emptyset$)。
\end{problem}

\begin{solution}
题意为判定 $L_1$ 与 $L_2$ 是否\textbf{有公共元素}(即 $L_1\cap L_2\ne\emptyset$)。

由于 $L_1$ 是 CFL,$L_2$ 是正则语言,且 CFL 与正则语言的交仍是 CFL,所以 $L_1\cap L_2$ 是 CFL。
而 CFL 的空性问题可判定(例如对 CFG 进行 generating/reachable 分析,或对 PDA 判定是否存在可接受路径)。

因此算法为:构造识别 $L_1\cap L_2$ 的 CFG/PDA,并判定其语言是否为空。非空则两者包含相同元素,空则不包含。
\end{solution}

\section*{CFL 非封闭性与 CFL 泵引理}

% --- Problem 5 ---
\begin{problem}
证明下述语言不是 CFL:
\[
K_1=\{a^n b^j\mid n=j^2\}.
\]
\end{problem}

\begin{solution}
用 CFL 泵引理(Bar-Hillel 引理)。假设 $K_1$ 是 CFL,设泵长度为 $p$。
取
\[
w=a^{p^2} b^{p}\in K_1.
\]
将 $w=uvwxy$,满足 $|vwx|\le p$、$|vx|\ge 1$ 且对所有 $i\ge 0$ 有 $uv^iwx^iy\in K_1$。
由于 $|vwx|\le p$,$vwx$ 至多覆盖 $b$ 段的一部分或跨越 $a/b$ 分界,但无法同时覆盖到“足够多的 $a$ 与 $b$”以维持二次关系。

分情况:
\begin{itemize}
  \item 若 $vx$ 全在 $b^p$ 中,则泵 $i=0$ 会减少 $b$ 的个数但不改变 $a$ 的个数,得到 $a^{p^2}b^{p-t}$。
  此时应满足 $p^2=(p-t)^2$,只能推出 $t=0$,与 $|vx|\ge 1$ 矛盾。
  \item 若 $vx$ 全在 $a^{p^2}$ 中,则泵 $i=2$ 会增加 $a$ 的个数但不改变 $b$ 的个数,得到 $a^{p^2+t}b^{p}$。
  此时应满足 $p^2+t=p^2$(因为 $b$ 仍是 $p$ 个,平方仍为 $p^2$),矛盾。
  \item 若 $vx$ 跨越分界,则泵会同时改变 $a$ 与 $b$ 的个数,但 $|vx|$ 与 $|vwx|$ 都至多为 $p$,因此 $b$ 的变化量至多为 $p$,
  使得 $b$ 从 $p$ 变为 $p+\Delta$($|\Delta|\le p$),而平方变化为
  \[
  (p+\Delta)^2 - p^2 = 2p\Delta + \Delta^2,
  \]
  其绝对值至少为 $2p-1$(取 $\Delta=\pm 1$ 时),远大于 $a$ 段可被改动的幅度(至多 $p$),因此不可能仍保持 $n=j^2$。
\end{itemize}
三种情况均导致矛盾,故 $K_1$ 不是 CFL。
\end{solution}

% --- Problem 6 ---
\begin{problem}
证明下述语言不是 CFL:
\[
K_2=\{a^n\mid n \text{ 为质数}\}.
\]
\end{problem}

\begin{solution}
用 CFL 泵引理,设泵长度为 $p$,取一个质数 $q>p$,令 $w=a^q\in K_2$。
任取分解 $w=uvwxy$ 满足泵条件。由于字母表单一,$v$ 与 $x$ 都是若干个 $a$,记 $|vx|=t\ge 1$。
泵 $i=q+1$,得到串长度
\[
|uv^{q+1}wx^{q+1}y| = q + q\cdot t = q(1+t),
\]
这是合数($q\ge 2$ 且 $1+t\ge 2$),因此不在 $K_2$,与泵引理矛盾。故 $K_2$ 不是 CFL。
\end{solution}

% --- Problem 7 ---
\begin{problem}
证明下述语言不是 CFL:
\[
K_3=\{ww^Rw\mid w\in\{a,b\}^*\}.
\]
\end{problem}

\begin{solution}
设反证 $K_3$ 是 CFL,泵长度为 $p$。取
\[
w_0=a^p b^p,\qquad
w=w_0 w_0^R w_0 = a^p b^p b^p a^p a^p b^p\in K_3.
\]
考虑任意分解 $w=uvwxy$ 且 $|vwx|\le p$。由于 $|vwx|$ 的长度限制,它只能落在 $w$ 的某个局部片段中,
无法同时跨越左右两侧的两处“镜像对应区域”。泵 $i=0$ 或 $i=2$ 将在某一局部改变字符个数/结构,
破坏“前后两段相同且中间为反转”的整体结构,从而使得泵后串不再能写成 $tt^Rt$ 的形式。
因此与 CFL 泵引理矛盾,$K_3$ 不是 CFL。
\end{solution}

% --- Problem 8 ---
\begin{problem}
用泵引理证明下述语言不是 CFL:
\[
\{0^n1^n0^n1^n\mid n\ge 0\},\qquad
\{0^n\#0^{2n}\#0^{3n}\mid n\ge 0\},\qquad
\{w\#t\mid w,t\in\{a,b\}^*,\ \text{且 } w \text{ 是 } t \text{ 的子串}\}.
\]
\end{problem}

\begin{solution}
下面给出每个语言的经典泵引理证明思路(均取 CFL 泵长度 $p$,构造一个“重复块足够长”的串,使 $|vwx|\le p$ 的局部泵动无法保持全局约束)。

\textbf{(1) $L=\{0^n1^n0^n1^n\}$ 非 CFL}

取 $w=0^p1^p0^p1^p$。任意 $|vwx|\le p$ 的片段只能落在至多两个相邻块中(无法同时影响四段计数)。
泵后将改变某一段或两段的长度,使得四段长度不再全相等,从而不在 $L$。

\textbf{(2) $L=\{0^n\#0^{2n}\#0^{3n}\}$ 非 CFL}

取 $w=0^p\#0^{2p}\#0^{3p}$。$|vwx|\le p$ 只能改变三个 $0$-块中的一个或跨越一个分隔符,
但泵后无法同时维持三个块的比例 $1:2:3$,矛盾。

\textbf{(3) $L=\{w\#t\mid w \text{ 是 } t \text{ 的子串}\}$ 非 CFL}

取 $w=a^p b^p \# a^p b^p$(左侧 $w$ 与右侧 $t$ 相同,显然满足“子串”)。
泵 $i=0$ 会在左侧(或右侧)的某个局部删除一段字符,导致左侧 $w$ 变成 $a^{p-r}b^p$ 或 $a^p b^{p-r}$,
而右侧 $t$ 仍为 $a^p b^p$(或结构被局部扰动),从而 $w$ 不可能再是 $t$ 的子串(因为 $t$ 只有一个从 $a$ 到 $b$ 的分界位置)。
因此与 CFL 泵引理矛盾。
\end{solution}

\end{document}
